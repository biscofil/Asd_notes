\chapter{NP Completezza}

Analizziamo la complessità ( o  "intrattabilità" ) di problemi:

Un problema $\mathcal{P} \subseteq \underbrace{\mathcal{I} }_\text{Istanze} x  \underbrace{\mathcal{S} }_\text{Soluzioni}$

Tipi di problemi:

\begin{enumerate}

\item Indecidibili: non esiste una soluzione algoritmica
\item Decidibili
\begin{enumerate}
\item Trattabili: quelli analizzati finora, tempo di esecuzione "ragionevole"
\item Intrattabili
\end{enumerate}

\end{enumerate}

\subsection{Esempio di algoritmo Indecidibile : Stop di Turing}

\myworries{Manca}

\subsection{Problemi decisionali e di ottimizzazione}

Un problema può essere di tipo decisionale se il tipo di output è booleano. Gli algoritmi studiati fino ad ora rientrano nella categoria dei problemi di ottimizzazione. Ci concentreremo ora sui problemi di tipo decisionale.

\subsection{Classi di complessità}

\subsubsection{Classe P}
Problemi risolvibili in tempo polinomiale

\begin{equation}
\{\mathcal{P} | \exists \text{ un algoritmo che risolve } \mathcal{P} \text{ in tempo polinomiale } O(n^k)\}
\end{equation}

\myworries{Manca schema}

\subsubsection{Classe NP}

\begin{equation}
\{\mathcal{P} | \exists \text{ un algoritmo di verifica per } \mathcal{P} \text{ di tempo polinomiale } O(n^k) \}
\end{equation}

\myworries{Manca schema}

\subsubsection{Complemento di un problema}

Il complemento $\mathcal{\overline{P}}$ del problema $\mathcal{P}$ è molto più complesso da calcolare in quanto, per verificarlo, si dovrebbero testare tutte le singole possibilità.

\myworries{Manca schema}

\subsubsection{Riducibilità polinomiale}

$\mathcal{P}_1 \underbrace{\leq_P}_\text{Riducibile polinomialmente} \mathcal{P}_2$ se esiste un algoritmo polinomiale che traforma ciascuna istanza de problema $\mathcal{P}_1$ in un'istanza del problema $\mathcal{P}_2$

\paragraph{Proprietà}

\myworries{Manca}

\subsubsection{Classe NPC - NP Completi}

\begin{equation}
\{\mathcal{P} | \mathcal{P} \in \text{ NP e } \forall \mathcal{P}' \in \text{ NP }, \mathcal{P}' \leq_P \mathcal{P} \}
\end{equation}

\myworries{Manca schema Venn}

\subsubsection{Teorema fondamentale dell'NP-Completezza}

\begin{equation}
P \cap NPC \neq \emptyset \Rightarrow P = NP
\end{equation}

\paragraph{Dimostrazione}

Ipotesi : $P \cap NPC \neq \emptyset$ \\
Tesi : $\text{P} = \text{NP}$ \\

Dimostrazione : 

\begin{enumerate}
\item $\text{P} \subseteq \text{NP}$, già dimostrata. OK
\item $\text{NP} \subseteq \text{P}$
\end{enumerate}

\myworries{Manca}
