\chapter{Heap}

{Heap: albero quasi completo con tutti i nodi dell'ultimo livello a sinistra}

{MaxHeap: La radice ha valore maggiore o uguale a quello dei figli}

{MinHeap: La radice ha valore minore o uguale a quello dei figli}


\begin{lemma}{Lemma 2}{theoexample}
Nell'array che rappresenta un heap di $n$ elementi, le foglie sono i nodi con indici che vanno alle posizioni

$\frac{n}{2}+1,\frac{n}{2}+2,\ldots,n$

\end{lemma}

\paragraph{Esempio:}

\begin{tikzpicture}[>=stealth, every node/.style={circle, draw, minimum size=0.75cm}]
\graph [tree layout, grow=down, fresh nodes, level distance=0.5in, sibling distance=0.5in]
    {
        16 -> { 
          15 -> { 14 -> { 2, 8 }, 7 -> {1} },
          10 -> { 9, 3 }
        } 
    };
\end{tikzpicture}

\begin{tabular}{|c|c|c|c|c|c|c|c|c|c|c|}
\hline
Elemento & 16 & 15 & 10 & 14 & 7 & 9 & 3 & 2 & 8 & 1 \\
\hline
Posizione & 1 & 2 & 3 & 4 & 5 & 6 & 7 & 8 & 9 & 10 \\
\hline
 &  &  &  &  & $\frac{n}{2}$ & $\frac{n}{2}+1$ & $\frac{n}{2}+2$ & $\frac{n}{2}+3$ & $\frac{n}{2}+4$ & $n$ \\
\hline
\end{tabular}

\paragraph{Lemma 3}

{Altezza di un nodo: cammino più lungo verso una foglia}

{Il teorema definisce la numerosità di nodi con una certa altezza:}

{Ci sono al massimo $\frac{n}{2^{h+1}}$ nodi di altezza $h$ in
un qualsiasi heap di $n$ elementi}

\paragraph{Max\_heapify}

{L'operazione max\_heapify permette di mantenere le proprietà di maxheap}

{Precondizioni:}

{Gli alberi binari con radice in left(i) e right(i) sono maxheap}

{Postcondizioni:}

{~~~~~~~~L'albero radicato in $i$ è un maxheap}

\lstinputlisting{code/max_heapify.txt}

{Tempo di esecuzione : $O(h)$ dove $h$ è l'altezza del nodi i poichè l'heap ha altezza $log(n)$ (LEMMA 2)}

\paragraph{Dato un vettore disordinato, costruire un heap}

\lstinputlisting{code/build_maxheap.txt}

{Invariante: ogni nodo in posizione $i+1,\ldots,n$ è radice di un maxheap, con n = MANCA}

{Sembrerebbe complessa $O(\frac{n}{2}\,log(n)) = O(n\,log(n))$ ma max\_heapify lavora principalmente su foglie, quindi la complessità è lineare $O(n)$.}

\begin{equation}
\sum_{h=0}^{log(n)}{\floor*{\frac{h}{2^{h-1}}}*O(h)} = O(n*\sum_{h=0}^{log(n)}{\floor*{\frac{h}{2^h}})}
\end{equation}

{L'ultima sommatoria è la serie nota}

\begin{equation}
\sum_{h=0}^{+\infty}{h*x^h} = \frac{x}{{(1-x)}^2}
\end{equation}

{Con $x=\frac{1}{2}$:}

\begin{equation}
\sum_{h=0}^{+\infty}{h*x^h} = \frac{\frac{1}{2}}{{(1-\frac{1}{2})}^2} = 2
\end{equation}

{Quindi}

\begin{equation}
O(n*\sum_{h=0}^{log(n)}{\floor*{\frac{h}{2^h}}}) = O(2n) = O(n)
\end{equation}

\section{Heapsort}

\lstinputlisting{code/heapsort.txt}

{INV = Il sottoarray che va dalla posizione $1$ alla posizione $i$ è un maxheap che contiene gli elementi più piccoli del intero vettore di partenza, mentre $A[a+1,\ldots,n]$ contiene gli $n-1$ elementi più grandi di $A[1..n]$ ordinati}{.}

\begin{teorema}{ }{theoexample}
L'algoritmo HeapSort ordina in loco $n$ elementi eseguendo nel peggiore dei casi $O(nlog(n))$ confronti in quanto algoritmo basato sui confronti.
\end{teorema}

\subsection{Code di priorità}

{{[}CLRS{]} pp. 135-140}

{Struttura dati che serve a mantenere un insieme dinamico i cui elementi ( aggiungibili e rimovibili ) hanno un valore associato detto chiave o peso.}

{Esistono due tipi di code di priorità:}

\begin{itemize}
\tightlist
\item
  {MaxPriorità (Si utilizza la struttura MaxHeap)}
\item
  {MinPriorità (Si utilizza la struttura MinHeap)}
\end{itemize}

{Le operazioni delle code di priorità massima/minima sono:}

{Insert( Coda s , Elemento X) : Inserisce l'elemento $X$ in $S$}

{Maximum( Coda s ): Restituisce l'elemento di $S$ con la chiave maggiore senza rimuoverlo}

{Minimum( Coda s) : Restituisce l'elemento di $S$ con la chiave minore senza rimuoverlo}

{Extract\_max( Coda s) : Elimina e restituisce l'elemento di $S$ con la chiave maggiore}

{Extract\_min( Coda s) : Elimina e restituisce l'elemento di $S$ con la chiave minore}

{Increase\_key( Coda s, Elemento x, Chiave k) : Incrementa il valore della chiave di $X$ al nuovo valore $K$. Si suppone che $K$ sia maggiore o uguale al valore corrente della chiave di $X$, $K \geq chiave(X)$}

{Decrease\_key( Coda s, Elemento x, Chiave k) : Decrementa il valore della chiave di $X$ al nuovo valore $K$. Si suppone che $K$ sia minore o uguale al valore corrente della chiave di $X$, $K \leq chiave(X)$}

\begin{center}\rule{0.5\linewidth}{\linethickness}\end{center}

\subsection{Implementazione di code di massima priorità tramite Heap}

\lstinputlisting{code/heap_maximum.txt}

{Complessità di Heap Maximum : $O(1)$}

\lstinputlisting{code/heap_extract_max.txt}

{Complessità di Heap Extract Max : $O(log(n))$}

\lstinputlisting{code/heap_increase_key.txt}

{Complessità di Heap Increase Key : $O(log(n))$}

{Invariante del while: L'array $A[i,\ldots,A.heapSize]$ soddisfa le proprietà di maxHeap tranne una possibile violazione: A{[}i{]} potrebbe essere più grande di A{[}parent(i){]}}

{Con un Heap di n elementi, la complessità è $O(log(n))$ in quanto il cammino dal nodo fino alla radice ha lunghezza }$O(log(n))$

\lstinputlisting{code/heap_insert.txt}

{Complessità di Heap\_Insert : }$O(log(n))$

{La ricerca su una cosa di priorità, nel caso peggiore, ha costo }$O(n)$

{Premessa : $1 \leq i \leq A.heapSize$}

\lstinputlisting{code/heap_delete.txt}

{Complessità di Heap Delete : }$O(log(n))$

\paragraph{Esercizio 1}

{Scrivere una funzione che determini se un albero è quasi completo. Gli output devono essere 1 se l'albero è quasi completo o 0 se non lo è. }{L'albero è rappresentato con notazione left e right.}

{Note: notiamo che non è sufficiente sapere se un albero è quasi completo o meno, necessitiamo anche di un valore per rappresentare l'albero completo.}

{Gli output saranno quindi: 0 - albero completo, 1 - albero quasi completo, 2 - albero non quasi completo.}

\lstinputlisting{code/is_quasi_completo.txt}

\lstinputlisting{code/is_quasi_completo_aux.txt}

{Complessità di is\_quasi\_completo : $O(n)$}

$T(n) = T(k) + T(n-k-1) + c = O(n)$

\paragraph{Esercizio 2}

{Siano dati due alberi binari completi di radice $r$ ed $s$ aventi la stessa altezza $h$ e dimensione totale (somma dei nodi dei due alberi) $n$. Le chiavi memorizzate nei nodi soddisfano la proprietà di maxheap. Si vuole creare un unico albero quasi completo maxheap, fusione dei due alberi, con altezza $h+1$ e dimensione $n$.}

{Le seguenti soluzioni vengono accettate:}

\begin{enumerate}
\tightlist
\item
  {Soluzione di costo e tempo $\Theta(n)$ e in spazio aggiuntivo $\Theta(n)$}
\item
  {Soluzione (più elegante) di costo e tempo $O(log(n))$ e spazio aggiuntivo costante}
\end{enumerate}

{Soluzione :}

\begin{enumerate}
\tightlist
\item
  {Soluzione di costo e tempo $\Theta(n)$ e in spazio aggiuntivo $\Theta(n)$}
\end{enumerate}

{~~~~~~~~~~~~~~~~~~~~~~~~Vettore di n elementi v. (spazio aggiuntivo $\Theta(n)$)}

{~~~~~~~~~~~~~~~~~~~~~~~~Carichiamo il vettore con i valori di $r$ ed $s$. $\Theta(n)$}

{~~~~~~~~~~~~~~~~~~~~~~~~Applichiamo build\_maxheap all'intero vettore $v$. $\Theta(n)$}

{~~~~~~~~~~~~~~~~~~~~~~~~Creo l'albero corrispondente. $\Theta(n)$}

\begin{enumerate}
\setcounter{enumi}{1}
\tightlist
\item
  {Soluzione di costo e tempo $O(log(n))$ e spazio aggiuntivo costante}
\end{enumerate}

{Non possiamo fare altro che modificare le strutture a nostra disposizione, non usiamo strutture ausiliarie.}

{Prendo come radice il nodo foglia $x$ più a destra dell'albero $s$. $\Theta(log(n))$}

{Assegno $x.left$ ad $s$ e $x.right$ ad $r$.}

{Applico la max\_heapify al nuovo nodo radice x. $\Theta(log(n))$}
